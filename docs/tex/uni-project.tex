\documentclass[12pt,a4paperб]{article}
\usepackage[utf8]{inputenc}
\usepackage[english, russian]{babel}
\usepackage{indentfirst}
\usepackage{misccorr}
\usepackage{graphicx}
\usepackage{amsmath}
\usepackage{parskip}
%\usepackage{tempora}
\usepackage[top=2cm, bottom=1cm, left=1cm, right=1cm]{geometry}
\usepackage[most]{tcolorbox}
\usepackage{hyperref}
\usepackage{fancyhdr}
\usepackage{svg}
\usepackage{multirow}
\usepackage{colortbl}

\hypersetup{
	colorlinks,
	citecolor=black,
	filecolor=black,
	linkcolor=blue,
	urlcolor=blue
}

\pagestyle{fancy}
\lhead{Университет}
\rhead{\thepage}
\cfoot{}

\graphicspath{{pic/}, {~/Pictures/TeXImgs/}, {~/Pictures/Joplin Icons}}

\newcommand{\sfrac}[2]{\dfrac{\strut #1}{\strut #2}}

\newcounter{picture}
\setcounter{picture}{1}
\newcounter{tbl}
\setcounter{tbl}{1}
\newcommand{\embed}[3]{\begin{center}
		\includegraphics[scale=#2]{#1}
		\\\textbf{Рис. \thepicture:} #3
		\label{pic_\thepicture}
		\addtocounter{picture}{1}
\end{center}}
\newcommand{\embedeps}[3]{\begin{center}
		\includegraphics[width=#2\linewidth]{#1}
		\\\textbf{Рис. \thepicture:} #3
		\label{pic_\thepicture}
		\addtocounter{picture}{1}
\end{center}}
\newcommand{\embedsvg}[3]{\begin{center}
		\includesvg[width=#2\linewidth]{#1}
		\\\textbf{Рис. \thepicture:} #3
		\label{pic_\thepicture}
		\addtocounter{picture}{1}
\end{center}}
\newcommand{\embedtbl}[2]{\begin{center}
		#1
		\textbf{Табл. \thetbl:} #2
		\label{tbl_\thetbl}
		\addtocounter{tbl}{1}
\end{center}}
\newcommand{\picref}[1]{\hyperref[pic_#1]{Рис. #1}}
\newcommand{\tblref}[1]{\hyperref[tbl_#1]{Табл. #1}}

\begin{document}
	
		\begin{titlepage}
		\vspace*{\fill}
		
		\begin{center}
			\includesvg[width=0.3\linewidth]{quarkus.svg}
			\\[0.5cm]\noindent\rule{\textwidth}{1pt}
			\\\Huge\textbf{Проектирование "университет"}
			\\[-0.5cm]\noindent\rule{\textwidth}{1pt}
		\end{center}
		
		\vspace*{\fill}
		
		\begin{flushleft}
			Выполнил: \hspace{\fill}
			\\Кошелев Александр \hspace{\fill}
		\end{flushleft}
	\end{titlepage}
	
	\tableofcontents
	
	\setcounter{page}{2}
	\newpage
	
	\section{Проектирование макета сервиса}
	
	\subsection{Постановка задачи}
	
	Задача в данном проекте ставится достаточно просто -- разработать бэкенд учетного сервиса.
	
	Необходимо иметь возможность оперировать как крупными структурами вроде факультета или направления обучения, так и мелкими, например, студентами. Также необходимо проанализировать ролевую систему в рамках приложения, проработать возможный вид UI и строить бэкенд методом "от потребителя".
	
	\subsection{Анализ ролевой модели}
	
	Поскольку система подразумевает самые разные уровни управления, все слои иерархии института должны получить соответствующие UI и функционал для ведения деятельности. Предполагаются следующие роли и полномочия:
	
	\begin{itemize}
		\item Ректорат -- просмотр и редактирование факультетов и направлений обучения.
		\item Дирекция факультета -- изменения в структурах курсов, предметов, пересмотр учебной части.
		\item Кадровая служба -- прием на работу преподавателей, изменения в кадровой структуре.
		\item Приемная комиссия -- прием на обучение студентов.
		\item Учебные управления факультетов -- пересмотр оценок студентов в исключительных случаях.
		\item Преподаватели -- редактирование оценок студентов.
		\item Студенты -- просмотр оценок и учебного плана.
	\end{itemize}
	
	На основании такой ролевой модели можно выделить необходимые для функционирования сервиса единицы GUI.  	 Представим их в табличном виде по отношению к имеющимся ролям и подробно опишем ниже.
	 
	 \embedtbl{
	 	\begin{table}[h]
	 		\centering
	 		\begin{tabular}{|
	 				>{\columncolor[HTML]{EFEFEF}}c |
	 				>{\columncolor[HTML]{FFFFC7}}c |
	 				>{\columncolor[HTML]{FFFFC7}}c |
	 				>{\columncolor[HTML]{FFFFC7}}c |
	 				>{\columncolor[HTML]{FFFFC7}}c |}
	 			\hline
	 			Роль/GUI &
	 			\cellcolor[HTML]{EFEFEF}\begin{tabular}[c]{@{}c@{}}Административная\\ структура\end{tabular} &
	 			\cellcolor[HTML]{EFEFEF}Персоналии &
	 			\cellcolor[HTML]{EFEFEF}\begin{tabular}[c]{@{}c@{}}Учебный\\ план\end{tabular} &
	 			\cellcolor[HTML]{EFEFEF}Журнал \\ \hline
	 			Ректорат &
	 			\cellcolor[HTML]{D9FFD8}EDIT &
	 			READ &
	 			READ &
	 			READ \\ \hline
	 			\begin{tabular}[c]{@{}c@{}}Дирекция\\ факультета\end{tabular} &
	 			READ &
	 			READ &
	 			\cellcolor[HTML]{D9FFD8}EDIT &
	 			READ \\ \hline
	 			\cellcolor[HTML]{EFEFEF} &
	 			\cellcolor[HTML]{FFFFC7} &
	 			\cellcolor[HTML]{D9FFD8}\begin{tabular}[c]{@{}c@{}}EDIT\\ Кадровая структура\end{tabular} &
	 			\cellcolor[HTML]{FFFFC7} &
	 			\cellcolor[HTML]{FFFFC7} \\ \cline{3-3}
	 			\multirow{-2}{*}{\cellcolor[HTML]{EFEFEF}\begin{tabular}[c]{@{}c@{}}Кадровая\\ служба\end{tabular}} &
	 			\multirow{-2}{*}{\cellcolor[HTML]{FFFFC7}READ} &
	 			\begin{tabular}[c]{@{}c@{}}READ\\ Остальное\end{tabular} &
	 			\multirow{-2}{*}{\cellcolor[HTML]{FFFFC7}READ} &
	 			\multirow{-2}{*}{\cellcolor[HTML]{FFFFC7}READ} \\ \hline
	 			\cellcolor[HTML]{EFEFEF} &
	 			\cellcolor[HTML]{FFFFC7} &
	 			\cellcolor[HTML]{D9FFD8}\begin{tabular}[c]{@{}c@{}}EDIT\\ Студенты\end{tabular} &
	 			\cellcolor[HTML]{FFFFC7} &
	 			\cellcolor[HTML]{FFFFC7} \\ \cline{3-3}
	 			\multirow{-2}{*}{\cellcolor[HTML]{EFEFEF}\begin{tabular}[c]{@{}c@{}}Приемная\\ комиссия\end{tabular}} &
	 			\multirow{-2}{*}{\cellcolor[HTML]{FFFFC7}READ} &
	 			\begin{tabular}[c]{@{}c@{}}READ\\ Остальное\end{tabular} &
	 			\multirow{-2}{*}{\cellcolor[HTML]{FFFFC7}READ} &
	 			\multirow{-2}{*}{\cellcolor[HTML]{FFFFC7}READ} \\ \hline
	 			\begin{tabular}[c]{@{}c@{}}Учебное\\ управление\end{tabular} &
	 			READ &
	 			READ &
	 			READ &
	 			\cellcolor[HTML]{D9FFD8}EDIT \\ \hline
	 			Преподаватели &
	 			READ &
	 			READ &
	 			READ &
	 			\cellcolor[HTML]{D9FFD8}EDIT \\ \hline
	 			Студенты &
	 			READ &
	 			READ &
	 			READ &
	 			READ \\ \hline
	 		\end{tabular}
	 	\end{table}
	 }{Права ролей}
	 
	 Начнем с \textit{административной структуры}, которая будет представлять из себя набор инструментов для просмотра и редактирования факультетов и направлений обучения. Далее \textit{персоналии} --- раздел с лицами института, преподавателями и студентами. \textit{Учебный план} будет набором инструментов планирования курсов, расписания и предметов с учетом распределения нагрузки и занятости преподавателей и студентов. Наконец, \textit{журнал} --- система учета успеваемости обучающихся.
	 
	 \subsection{Макеты UI}
	 
	 Прикинем примерный интерфейс соответствующих инструментов.
	 
	 \subsubsection{Административная структура}
	 
	 \embedsvg{admin.svg}{0.7}{UI административная структура}
	 
	 \begin{enumerate}
	 	\item Navigation drawer -- навигация по сайту (не требует бек)
	 	\item ImageButton -- переход в личный кабинет (требует информацию о пользователе)
	 	\item Toolbar -- инструменты фильтрации и настройки глубины графа (требует фильтры на сущности)
	 	\item Graph -- рабочее поле с возможностью перехода на страницы соответствующих звеньев (требует данные о сущностях)
	 \end{enumerate}
	 
	 \subsubsection{Персоналии}
	 
	 \embedsvg{people.svg}{0.7}{UI персоналии}
	 
	 \begin{enumerate}
	 	\item Navigation drawer -- навигация по сайту (не требует бек)
	 	\item ImageButton -- переход в личный кабинет (требует информацию о пользователе)
	 	\item Toolbar -- инструменты фильтрации  и выборки (требует фильтры на сущности)
	 	\item List -- рабочее поле с возможностью перехода на страницы соответствующих персон (требует данные о сущностях)
	 \end{enumerate}
	 
	 \subsubsection{Учебный план}
	 
	 \embedsvg{plan.svg}{0.7}{UI учебный план}
	 
	 \begin{enumerate}
	 	\item Navigation drawer -- навигация по сайту (не требует бек)
	 	\item ImageButton -- переход в личный кабинет (требует информацию о пользователе)
	 	\item Toolbar -- инструменты фильтрации и выборки (требует фильтры на сущности)
	 	\item Table -- рабочее поле, сводная таблица учебного плана (требует данные о сущностях)
	 \end{enumerate}
	 
	 	 \subsubsection{Журнал}
	 
	 \embedsvg{plan.svg}{0.7}{UI журнал}
	 
	 \begin{enumerate}
	 	\item Navigation drawer -- навигация по сайту (не требует бек)
	 	\item ImageButton -- переход в личный кабинет (требует информацию о пользователе)
	 	\item Toolbar -- инструменты фильтрации и выборки (требует фильтры на сущности)
	 	\item Table -- рабочее поле, таблица успеваемости студентов в группе  (требует данные о сущностях)
	 \end{enumerate}
	 
	 \section{Проектирование API}
	 
	 \subsection{Выявление достаточных DTO}
	 
	 Рассмотрим, какая информация необходима нашему UI для его работы, и отобразим, какие поля должны быть у наших DTO, чтобы эту информацию доставить.
	 
	  \subsubsection{Административная структура}
	  
	 	 \embedsvg{admin_dto.svg}{0.7}{DTO административная структура} 
	 
\end{document}